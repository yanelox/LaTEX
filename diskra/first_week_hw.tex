\documentclass[12pt,a4paper,fleqn]{article}
\usepackage[utf8]{inputenc}
\usepackage{amssymb, amsmath, multicol}
\usepackage[russian]{babel}
\usepackage{graphicx}
\usepackage[shortcuts,cyremdash]{extdash}
\usepackage{wrapfig}
\usepackage{floatflt}
\usepackage{lipsum}
\usepackage{concmath}
\usepackage{euler}

\oddsidemargin=-15.4mm
\textwidth=190mm
\headheight=-32.4mm
\textheight=277mm
\tolerance=100
\parindent=0pt
\parskip=8pt
\pagestyle{empty}
\renewcommand{\tg}{\mathop{\mathrm{tg}}\nolimits}
\renewcommand{\ctg}{\mathop{\mathrm{ctg}}\nolimits}
\renewcommand{\arctan}{\mathop{\mathrm{arctg}}\nolimits}
\newcommand{\divisible}{\mathop{\raisebox{-2pt}{\vdots}}}


\begin{document}

{\bf 1)} $ \land, \lor, \equiv, +, |, \downarrow$ ~---~ коммутативны, что легко доказывается путем построения таблицы истинности для обоих вариантов выражения. Докажем это на примере $ \land $, для остальных все будет аналогично.\newline


\begin{tabular}{| c | c | c |}   
\hline
x & y & F \\ \hline
0 & 0 & 0 \\
0 & 1 & 0 \\
1 & 0 & 0 \\
1 & 1 & 1\\
\hline
\end{tabular} 
\quad
\begin{tabular}{| c | c | c |}   
\hline
y & x & F \\ \hline
0 & 0 & 0 \\
0 & 1 & 0 \\
1 & 0 & 0 \\
1 & 1 & 1\\
\hline
\end{tabular}\newline


{\bf 2)} Построим таблицы истинности для выражений $x \land (y \land z)$ и 
\newline
$(x \land y) \land z$ и заметим, что они равносильны \newline


\begin{tabular}{| c | c | c | c | c |}   
\hline
x & y & z & $y \land z$ & F\\ \hline
0 & 0 & 0 & 0 & 0 \\
0 & 0 & 1 & 0 & 0 \\
0 & 1 & 0 & 0 & 0 \\
0 & 1 & 1 & 1 & 0 \\
1 & 0 & 0 & 0 & 0 \\
1 & 0 & 1 & 0 & 0 \\
1 & 1 & 0 & 0 & 0 \\
1 & 1 & 1 & 1 & 1 \\
\hline
\end{tabular}
\quad
\begin{tabular}{| c | c | c | c | c |}   
\hline
x & y & z & $x \land y$ & F\\ \hline
0 & 0 & 0 & 0 & 0 \\
0 & 0 & 1 & 0 & 0 \\
0 & 1 & 0 & 0 & 0 \\
0 & 1 & 1 & 0 & 0 \\
1 & 0 & 0 & 0 & 0 \\
1 & 0 & 1 & 0 & 0 \\
1 & 1 & 0 & 1 & 0 \\
1 & 1 & 1 & 1 & 1 \\
\hline
\end{tabular} \newline


{\bf 3a)} Построим таблицу истинности для выражения $x \lor (x \land y)$, чтобы доказать утверждение и заметим что F = x \newline


\begin{tabular}{| c | c | c | c |}
\hline
x & y & $x \land y$ & F \\ \hline
0 & 0 & 0 & 0 \\
0 & 1 & 0 & 0 \\
1 & 0 & 0 & 1 \\
1 & 1 & 1 & 1 \\ 
\hline
\end{tabular} \newline


{\bf 3b)} Разложим оператор импликации через оператор дизъюнкции:
\newline $\overline{x\rightarrow y} = \overline{\overline{x} \lor y} = x\overline{y}$. \newline


{\bf 4)} \quad \quad \quad $x1\overline{x2} \lor x2\overline{x3}$
\quad \quad \quad \quad \quad \quad \quad \quad \quad \quad 
$(x1 \lor x2)(\overline{x2} \lor x3)$ 


\begin{tabular}{| c | c | c | c | c | c|}
\hline
   &    &    &              &              &   \\
x1 & x2 & x3 & $x1\overline{x2}$ & $x2\overline{x3}$ & F \\ \hline
0 & 0 & 0 & 0 & 0 & 0 \\ 
0 & 0 & 1 & 0 & 0 & 0 \\
0 & 1 & 0 & 0 & 1 & 1 \\
0 & 1 & 1 & 0 & 0 & 0 \\
1 & 0 & 0 & 1 & 0 & 1 \\
1 & 0 & 1 & 1 & 0 & 1 \\
1 & 1 & 0 & 0 & 1 & 1 \\
1 & 1 & 1 & 0 & 0 & 0 \\
\hline
\end{tabular}
\quad
\begin{tabular}{| c | c | c | c | c | c|}
\hline
   &    &    &              &              &   \\
x1 & x2 & x3 & $x1 \lor x2$ & $\overline{x2} \lor x3$ & F \\ \hline
0 & 0 & 0 & 0 & 1 & 0 \\ 
0 & 0 & 1 & 0 & 1 & 0 \\
0 & 1 & 0 & 1 & 0 & 0 \\
0 & 1 & 1 & 1 & 1 & 1 \\
1 & 0 & 0 & 1 & 1 & 1 \\
1 & 0 & 1 & 1 & 1 & 1 \\
1 & 1 & 0 & 1 & 0 & 0 \\
1 & 1 & 1 & 1 & 1 & 1 \\
\hline
\end{tabular} 	\newline
				\newline
Построив таблицы, получаем что первая функция не равна второй \newpage

{\bf 5a)} $(x + y) \lor \overline{y}$ ~---~  построив таблицу, можем увидеть, что x, y - существенные \newline 

\begin{tabular}{| c | c | c | c |}
\hline
x & y & x + y & F \\ \hline
0 & 0 & 0 & 1\\
0 & 1 & 1 & 1\\
1 & 0 & 1 & 1\\
1 & 1 & 0 & 0\\
\hline
\end{tabular} \newline

{\bf 5b)} $(x \rightarrow (x \lor y)) \rightarrow z$ ~---~  построив таблицу, можем увидеть, что F = z, поэтому x, y - несущественные, z - существенная\newline

\begin{tabular}{| c | c | c | c | c | c |}
\hline
x & y & z & $x \lor y$ & $x \rightarrow (x \lor y)$ & F \\ \hline
0 & 0 & 0 & 0 & 1 & 0 \\
0 & 0 & 1 & 0 & 1 & 1 \\
0 & 1 & 0 & 1 & 1 & 0 \\
0 & 1 & 1 & 1 & 1 & 1 \\
1 & 0 & 0 & 1 & 1 & 0 \\
1 & 0 & 1 & 1 & 1 & 1 \\
1 & 1 & 0 & 1 & 1 & 0 \\
1 & 1 & 1 & 1 & 1 & 1 \\
\hline
\end{tabular} \newline

{\bf 6)} $(x \rightarrow y) \rightarrow (\overline{y} \rightarrow \overline{x})= (\overline{x} \lor y) \rightarrow (y \lor \overline{x}) = \overline{\overline{x} \lor y} \lor y \lor \overline{x} = x\overline{y} \lor \overline{x\overline{y}} = A \lor \overline{A} = 1 \quad (A = x\overline{y})$ \newline

{\bf 7)} $x \land y = \overline{x | y}$, $x \lor y = \overline{x} | \overline{y}$, $x \equiv y = \overline{x | y} \lor \overline{\overline{x} | \overline{y}}$, $x + y = (x | y)(\overline{x} | \overline{y})$, $x \downarrow y = \overline{\overline{x} | \overline{y}}$

\end{document}