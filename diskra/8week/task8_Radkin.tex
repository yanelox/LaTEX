\documentclass[12pt,a4paper,fleqn]{article}
\usepackage[utf8]{inputenc}
\usepackage{amssymb, amsmath, multicol}
\usepackage[russian]{babel}
\usepackage{graphicx}
\usepackage[shortcuts,cyremdash]{extdash}
\usepackage{wrapfig}
\usepackage{floatflt}
\usepackage{lipsum}
\usepackage{concmath}
\usepackage{euler}
\usepackage{tikz}  
\usetikzlibrary{graphs}

\oddsidemargin=-15.4mm
\textwidth=190mm
\headheight=-32.4mm
\textheight=277mm
\tolerance=100
\parindent=0pt
\parskip=8pt
\pagestyle{empty}
\renewcommand{\tg}{\mathop{\mathrm{tg}}\nolimits}
\renewcommand{\ctg}{\mathop{\mathrm{ctg}}\nolimits}
\renewcommand{\arctan}{\mathop{\mathrm{arctg}}\nolimits}
\newcommand{\divisible}{\mathop{\raisebox{-2pt}{\vdots}}}

\begin{document}
{\bf 1a)} $C_n^k = C_{n - 1}^{k - 1} + C_{n - 1}^k \rightarrow$ \newline
$C_{n + k + 1}^k = C_{n + k}^{k - 1} + C_{n + k}^k = C_{n + k - 1}^{k - 2} + C_{n + k - 1}^{k - 1} + C_{n + k}^k \rightarrow$ Раскладывая таким образом слагаемые, мы будем получать элементы суммы $\sum_{m = 0}^k C_{n + m}^m$. В самом конце останется слагаемое $C_{n + 2}^1 = C_{n + 1}^0 + C_{n + 1}^1 = 1 + C_{n + 1}^1 = C_n^0 + C_{n + 1}^1$. Ч.~т.~д. \newline \newline
{\bf 1б)} $C_n^k = \dfrac{n!}{(n - k)!k!} \rightarrow$ Т.~к. $n = const$, найдем минимум функции $(n -k)!k!$. Если мы увеличиваем $k$ на единицу, тогда наша функция уменьшается в $\dfrac{n - k}{k - 1}$. Решая неравенство $\dfrac{n - k}{k - 1} > 1 \rightarrow k < \dfrac{n}{2} - 1 \rightarrow k = \dfrac{n}{2}$~---~точка минимума функции и, соответственно возрастает при меньших $k$ и убывает при больших $k$.\newline
Числа $F_{998}$ и $F_{999}$ равноудалены от $\dfrac{F_{1000}}{2}$, т.~к. $F_{999} - \dfrac{F_{1000}}{2} = F_{1000} - F_{998} - \dfrac{F_{1000}}{2} = \dfrac{F_{1000}}{2} - F_{998}$, но, т.~к. они оба увеличены на единицу, первое число приблизилось к середине, а второе наоборот удалилось, поэтому, больше будет число $C_{F_{1000}}^{F_{998} + 1}$ \newline
Ответ: $C_{F_{1000}}^{F_{998} + 1}$ \newline \newline
{\bf 1в)} $x_k = C_n^k \cdot 2^k, x_{k + 1} = C_n^{k + 1} \cdot 2^{k + 1} \rightarrow x_{k + 1} = \dfrac{n!}{(n - k - 1)!(k + 1)!}2^{k + 1} = \dfrac{n!}{(n - k)!k!}2^k \cdot \dfrac{2(n - k)}{k + 1} = x_k \cdot \dfrac{2(n - k)}{k + 1} \rightarrow$ \newline
$\dfrac{2(n - k)}{k + 1} \geqslant 1 \rightarrow k \leqslant \dfrac{2n - 1}{3} \rightarrow$ Т.~к. $\dfrac{2n - 1}{3}$ - последнее число $k$, следующий элемент после которого больше его самого $\rightarrow$ мы должны выбрать $k = \dfrac{2n + 2}{3}$ (во всех формулах подразумевается целая часть от числа).\newline
Тогда, максимальный член: $C_n^{\frac{2n + 2}{3}} \cdot 2^{\frac{2n + 2}{3}}$ \newline
Ответ: $C_n^{\frac{2n + 2}{3}} \cdot 2^{\frac{2n + 2}{3}}$ \newline \newline
{\bf 1г)} $x_n = \dfrac{2^{2n}}{n + 1}$ 
\begin{itemize} 
\item База индукции, $n = 2$: $C_4^2 = \dfrac{4!}{2!2!} = 6$, $\dfrac{2^4}{3} = \dfrac{16}{3} < 6$~---~верно.
\item Шаг индукции: $C_{2n + 2}^{n + 1} = \dfrac{(2n + 2)!}{(n + 1)!(n + 1)!} = \dfrac{(2n + 2)(2n + 1)}{(n + 1)^2} \cdot\dfrac{(2n)!}{n!n!} = \dfrac{(2n + 2)(2n + 1)}{(n + 1)^2} \cdot C_{2n}^n$ \newline
$\dfrac{2^{2n + 2}}{n + 2} = \dfrac{2^{2n}(n + 1) \cdot 4}{(n + 1)(n + 2)} = x_n \cdot \dfrac{4n + 4}{n + 2}$ \newline
Проверим, что: $\dfrac{(2n + 2)(2n + 1)}{(n + 1)^2} > \dfrac{4n + 4}{n + 2}$ \newline
$4n^3 + 6n^2 + 2n + 8n^2 + 12n + 4 > 4n^3 + 12n^2 + 12n + 4$ \newline
$14n^2 + 2n > 12n^2$ \newline
$2n^2 + 2n > 0$~---~верно $\rightarrow$ т.~к. $C_{2n}^n > x_n$ по предположению индукции $\rightarrow$ $C_{2n + 2}^{n + 1} > x_{n + 1}$, ч.~т.~д. \newline
\end{itemize}
{\bf 2а)}
\begin{itemize}
\item Посчитаем кол-во чисел, не содержащих единицу. Тогда, на любое из 6-ти мест мы может поставить $9$ различных цифр $\rightarrow$ кол-во вариантов $9^6 = 531441 > 500000 \rightarrow$ не содержащих единицу больше.
\item Аналогично для 10-ти миллионов: $9^7 = 4782969 < 5000000 \rightarrow$ содержащих единицу больше.
\end{itemize}
{\bf 2б)}
\begin{itemize}
\item Если на первом месте четная цифра ($4$ возможных варианта), на остальных местах по 5 возможных вариантов $\rightarrow 4 \cdot 5^5$. Кроме того, учтем кол-во перестановок~---~$C_5^2$(как переставить $2$ нечетных числа по $5$ местам). $\rightarrow$ $4 \cdot 5^5 \cdot C_5^2$
\item Если на первом месте нечетная цифра ($5$ возможных вариантов), остальное аналогично $\rightarrow$ $5 ^ 6 \cdot C_5^2$
\end{itemize}
$\rightarrow 9 \cdot 5^5 \cdot C_5^2$ \newline
Ответ: $9 \cdot 5^5 \cdot C_5^2$ \newline \newline
{\bf 2в)} Т.~к. четных и нечетных цифр поровну ($5$), то на каждое место мы можем поставить по $5$ различных цифр. Остается учесть перестановки четных. Сначала мы выбираем $1$ место среди $6$, потом $1$ среди $4$ (т.~к. перед каждым четным~---~нечетное). Таким образом, получаем: $5^7 \cdot C_6^1 \cdot C_4^1 = 24 \cdot 5^7$ \newline
Ответ: $24 \cdot 5^7$ \newline \newline
{\bf 3)}
\begin{itemize}
\item $A \geqslant 5$: Наш путь не поднимется выше 5, т.~к. иначе мы не сможем попасть в $(10, 0)$. Таким образом, кол-во способов: $C_{10}^5$
\item $A < 5$. Тогда, путь может пересечь прямую $y = A$. Тогда, найдем первую точку пересечения для всех таких путей и отразим весь путь после этой точки относительно $y = A$, тогда наш путь теперь приходит в точку $(10, 2A)$. Тогда, чтобы попасть в эту точку, нам нужно выбрать $a$ шагов вверх, тогда: $a - (10 - a_ = 2A \rightarrow a = 5 + A \rightarrow C_{10}^{5 + A} \rightarrow$ получаем $C_{10}^5 - C_{10}^{5 + A}$
\end{itemize} 
Ответ: при $A \geqslant 5$: $C_{10}^5$, при $A < 5$: $C_{10}^5 - C_{10}^{5 + A}$ \newline \newline
{\bf 4)}
\begin{itemize} 
\item Найдем кол-во путей, которые пересекают $y = 6$: для этого найдем первую точку пересечения нашего пути с $y = 6$ и отразим последующий путь относительно $y = 6$, получаем путь, приходящий в точку $(20, 12)$. Тогда, таких путей: $C_{20}^{16}$ 
\item Аналогично, кол-во путей, уходящих под $y = -6$~---~$C_{20}^{16}$
\item Кроме того, подсчитаем кол-во путей, которые пересекают обе прямые $y = 6$ и $y = -6$. Чтобы такое произошло, путь должен сделать как минимум $6$ шагов вверх, потом $12$ шагов вниз, потом $6$ шагов вверх. $6 + 6 + 12 = 24 > 20 \rightarrow$ это невозможно.
\item Всего путей из $(0, 0)$ в $(20, 0)$: $C_{20}^{10}$
\end{itemize}
Таким образом, мы должны исключить все пути, выходящие за пределы: $C_{20}^{10} - 2 \cdot C_{20}^{16}$ \newline
Ответ: $C_{20}^{10} - 2 \cdot C_{20}^{16}$ \newline \newline
{\bf Бонусная задача)} $(0, 0) \rightarrow (n, k)$. Очевидно, что при четных $n$, $k$~---~ четные, и наоборот
\begin{itemize}
\item $n$~---~четные. Тогда, пусть нам нужно выбрать $a$ шагов вверх, тогда: $a - (n - a) = k \rightarrow a = \dfrac{n + k}{2} \rightarrow C_{n}^{\frac{n + k}{2}}$~---~кол-во вариантов для одного $k$. Тогда, для всех $k$: $\sum_{k = 0}^{\frac{n}{2}} C_n^{\frac{n + 2k}{2}}$
\item $n$~---~нечетное, все рассуждения аналогичны, только сумма в итоге будет считаться только по нечетным $k$: $\sum_{k = 0}^{\frac{n - 1}{2}} C_n^{\frac{n + 2k + 1}{2}}$
\end{itemize}
Теперь заметим, что: $\sum_{k = 0}^{\frac{n}{2}} C_n^{\frac{n + 2k}{2}} + \sum_{k = 0}^{\frac{n - 1}{2}} C_n^{\frac{n + 2k + 1}{2}} = \sum_{k = 0}^n C_n^{\frac{n + k}{2}}$ \newline
Ответ: $\sum_{k = 0}^n C_n^{\frac{n + k}{2}}$ \newpage
{\bf 5)} 
\begin{itemize}
\item Пусть, мы разбили число $N$ на $a$ слагаемых, тогда имеем формулу: $\lambda_1 + \lambda_2 + \cdots + \lambda_a = N$ \newline
Тогда: $N + k = \lambda_1 + \lambda_2 + \cdots + \lambda_a + k$ \newline
Тогда, мы можем представить $k$ как сумму $k$ единиц, и $a$ из них прибавить к $\lambda_i$, т.~е. каждое $\lambda_i$ увеличится на единицу. Остальные $k -a$ единиц мы оставляем, и, таким образом получаем искомое разбиение числа $N + k$ на $k$ слагаемых, удовлетворяющих условию. Это означает, что разбиений числа $N + k$ на $k$ слагаемых как минимум не меньше, чем разбиений $N$ на любое число слагаемых, не большее $k$. 
\item Теперь, построим разбиение $N + k = \lambda_1 + \lambda_2 + \cdots + \lambda_k$ \newline
$N = \lambda_1 + \lambda_2 + \cdots + \lambda_k - k = (\lambda_1 - 1) + (\lambda_2 - 1) + \cdots + (\lambda_k - 1)$ \newline
Таким образом, мы построили разбиение числа $N$ на $a$ слагаемых ($a \leqslant k$, т.~к. некоторые слагаемые могут занулиться). Таким образом, число разбиений числа $N$ на $a$ слагаемых ($a \leqslant k$) не меньше, чем число разбиений числа $N + k$ на $k$ слагаемых.
\end{itemize}
Таким образом, мы получили, что каждое из количеств разбиений не меньше другого, что, фактически, означает их равенство. \newline
Ответ: одинаково \newline \newline
{\bf 6а)} $\displaystyle \binom{n}{m}\binom{m}{k}$~---~кол-во способов для того чтобы разбить мн-во размером $n$ на подмножества размером $k, m - k, n - m$. Сначала выделяем $m$ элементов и получаем два подмножества $m$ и $n - m$, потом подмножество $m$ также делим на два подмножества $k$ и $m - k$. \newline
$\displaystyle \binom{n}{k}\binom{n - k}{m - k}$~---~сначала разбиваем на $k$  и $n - k$. Потом, $n - k$ разбиваем на $m - k$ и $n - k -m + k = n - m$. Получаем аналогичное разбиение на $3$  подмножества размерами $k$, $n - m$, $m - k$ \newline
Таким образом, первое и второе выражение показывают кол-во способов одного и того же разбиения $\rightarrow$ они равны, ч.~т.~д. \newline \newline
{\bf 6б)} $\displaystyle \binom{n}{m}$~---~кол-во способов выделить из мн-ва размером $n$ подмножество размером $m$. \newline
Если мы уберем из мн-ва размером $n$ два элемента, то у нас возможно три случая:
\begin{itemize}
\item Если оба выброшенных элемента буду входить в мн-во размером $m$, тогда, из мн-ва размером $n -2$ нам нужно выбрать $m - 2$ элемента и потом добавить к нему оба выброшенных элемента $\rightarrow \displaystyle \binom{n - 2}{m - 2}$
\item Если только один из выброшенных будет входить, тогда нам необходимо выбрать $m - 1$ из $n - 2$, и добавить либо первый, либо второй элемент $\rightarrow 2 \displaystyle \binom{n - 2}{m - 1}$
\item Если ни один из элементов не будет входить в $m$, тогда нам нужно выбрать  $m$ из $n - 2$ $\rightarrow \displaystyle \binom{n - 2}{m}$
\end{itemize}
Таким образом, кол-во способов составить подмножество размером $m$ путем выбрасывания двух элементов из мн-ва размером $n$ это $\displaystyle \binom{n -2 }{m - 2} + 2\binom{n - 2}{m - 1} + \binom{n - 2}{m} = \binom{n}{m}$ по определению, ч.~т.~д.
\end{document}
