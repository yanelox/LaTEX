\documentclass[12pt,a4paper,fleqn]{article}
\usepackage[utf8]{inputenc}
\usepackage{amssymb, amsmath, multicol}
\usepackage[russian]{babel}
\usepackage{graphicx}
\usepackage[shortcuts,cyremdash]{extdash}
\usepackage{wrapfig}
\usepackage{floatflt}
\usepackage{lipsum}
\usepackage{concmath}
\usepackage{euler}
\usepackage{tikz}  
\usetikzlibrary{graphs}

\oddsidemargin=-15.4mm
\textwidth=190mm
\headheight=-32.4mm
\textheight=277mm
\tolerance=100
\parindent=0pt
\parskip=8pt
\pagestyle{empty}
\renewcommand{\tg}{\mathop{\mathrm{tg}}\nolimits}
\renewcommand{\ctg}{\mathop{\mathrm{ctg}}\nolimits}
\renewcommand{\arctan}{\mathop{\mathrm{arctg}}\nolimits}
\newcommand{\divisible}{\mathop{\raisebox{-2pt}{\vdots}}}

\begin{document}
{\bf 11)} Выберем некоторое произвольное поднможество $A$, состоящее из $k$ элементов. Тогда, посчитаем, сколько подмножеств $B$, которые с ним не пересекаются. Рассчитаем это таким образом: у нас осталось $n - k$ элементов, которые не приналежат $A$, нам нужно выбраь произвольное количество элементов из этой группы, чтобы получить не пересекающееся с ним подмножество $B$ $\rightarrow$ $\sum_{b = 0}^{n - k}C_{n - k}^b = 2^{n - k}$ (по биному Ньютона). \newline
Для каждого $k$, очевидно, существует $C_n^k$ множеств $A$ $\rightarrow$ существует $C_n^k \cdot 2^{n - k}$ пар непересекающихся множеств для одного из $k$. Тогда, для всех: $\sum_{k = 0}^{n}C_n^k \cdot 2^{n - k} = \sum_{k = 0}^{n}C_n^{n - k} \cdot 2^{n - k} = \sum_{k = 0}^{n}C_n^k \cdot 2^k = 3^n$ (по биному Ньютона). Также, необходимо разделить на 2, т.~к. каждую пару мы учли дважды $\rightarrow$ $\dfrac{3^n}{2}$ \newline
Всего подмножеств $2^n$ $\rightarrow$ кол-во способов выбрать пару подмножеств: $\dfrac{2^n(2^n - 1)}{2}$ (делим на 2, т.~к. каждую пару учли дважды). Тогда, искомая вероятность: $\dfrac{3^n \cdot 2}{2 \cdot 2^n \cdot (2^n - 1)} = \dfrac{3^n}{2^n(2^n - 1)}$ \newline
Ответ: $\dfrac{3^n}{2^n(2^n - 1)}$
\end{document}