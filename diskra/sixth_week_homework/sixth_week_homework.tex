\documentclass[12pt,a4paper,fleqn]{article}
\usepackage[utf8]{inputenc}
\usepackage{amssymb, amsmath, multicol}
\usepackage[russian]{babel}
\usepackage{graphicx}
\usepackage[shortcuts,cyremdash]{extdash}
\usepackage{wrapfig}
\usepackage{floatflt}
\usepackage{lipsum}
\usepackage{concmath}
\usepackage{euler}
\usepackage{tikz}  
\usetikzlibrary{graphs}

\oddsidemargin=-15.4mm
\textwidth=190mm
\headheight=-32.4mm
\textheight=277mm
\tolerance=100
\parindent=0pt
\parskip=8pt
\pagestyle{empty}
\renewcommand{\tg}{\mathop{\mathrm{tg}}\nolimits}
\renewcommand{\ctg}{\mathop{\mathrm{ctg}}\nolimits}
\renewcommand{\arctan}{\mathop{\mathrm{arctg}}\nolimits}
\newcommand{\divisible}{\mathop{\raisebox{-2pt}{\vdots}}}

\begin{document}
\begin{center}
Раздел 5:
\end{center}
{\bf 1)} Ответ~---~нет, очевидный контрпример~---~граф-треугольник. \newline
\begin{tikzpicture}
	[scale=.8,auto=left,every node/.style={circle,fill=blue!20}]
 	\node (n1) at (0, 0)  	{2};
  	\node (n2) at (4, 0)  	{2};
  	\node (n3) at (2, 2)  	{2};

  	\foreach \from/\to in {n1/n2, n2/n3, n1/n3}
    \draw (\from) -- (\to);
\end{tikzpicture} \newline
{\bf 3)} 3 вершины степени 1~---~<<начало дерева>> и 2 листа. Рассуждение следующее: т.~к. двигаясь из вершины дерева, мы должны иметь путь в оба листа, следовательно, есть хотя бы одна вершина со степенью, большей, чем 2 (чтобы получилось разветвление). Если эта вершина имела бы степень больше, чем 3, то в нашем дереве было бы больше листьев, следовательно у нас только вершины степени 3. Аналогично, не может быть больше одной вершины степени 3. \newline
Ответ:1 вершина \newline
{\bf 4)} Выберем $n=2$, тогда $d=1$. Очевидно, что для таких деревьев условие невыполнимо \newline
Ответ: Нет \newline
{\bf 5)} Построим полный граф, такой, что все его вершины имею степень $d$ ($\rightarrow$ всего в нем $d+1$ вершин). Тогда, очевидно, такой граф $d+1$ раскрашиваемый. Тогда, чтобы построить любой граф, степени вершин которого не больше $d$, достаточно:
\begin{itemize}
\item Убрать ребра/вершины из текущего графа. Очевидно, что полученный граф все еще $d+1$ раскрашиваемый.
\item Добавить новые вершины. Тогда, мы можем соединить их не более чем с $d$ вершинами, тогда они все, в крайнем случае, окрашены в $d$ цветов, тогда мы окрашиваем новую вершину в оставшийся $d+1$-ый цвет. В случае, если цвета некоторых вершин, с которыми мы соединяемся, совпадают, выбор для раскрашивания нашей новой вершины еще больше.
\end{itemize} 
{\bf 8a)} Для начала раскрасим наш основной цикл, тогда у нас каждый элемент $v_k$ окрашен в цвет 1, а $v_{k+1}$~---~в цвет 2 ($k$ - нечетное число, т.е. все нечетные вершины окрашены в цвет 1, все четные~---~в цвет 2). Тогда, мы должны соединить $v_k$ и $v_{k+n}$. Если $n$~---~четное, то вершины $v_k$ и $v_{k+n}$ равной четности $\rightarrow$ окрашены в один цвет $\rightarrow$ граф не двураскрашиваемый. В случае нечетного $n$, с раскраской все в порядке. \newline
Ответ: для любых нечетных $n$ \newline
{\bf 8b)} Очевидное замечание: если граф $n$-раскрашиваемый, то он $n+k$ раскрашиваемый, где $k>0$. Таким образом, т.~к. мы доказали возможность раскраски двумя цветами для нечетных $n$, следовательно, мы доказали и возможность раскраски тремя цветами(замечание 1). Осталось доказать для четных $n$. Для начала, аналогично, окрашиваем цикл в два цвета по принципу: $v_k$ в цвет 1 для нечетных $k$ и в цвет 2 для четных $k$. Теперь, мы соединяем $v_i$ и $v_{i+n}$, очевидно, что они окрашены в один цвет($i\leqslant n$). Теперь перекрасим некоторые точки по следующему правилу: для пары $v_i$ и $v_{i+n}$ окрашиваем первую точку в цвет 3, если $i$~---~нечетное, и, соответственно вторую точку в цвет 3, если $i$~---~четное. Таким образом, в <<первой половине>> графа у нас в цвет 3 окрашены все точки с нечетными номерами, а во <<второй половине>>~---~все точки с четными номерами, поэтому у нас никакие две соседние точки не окрашены в одинаковый цвет (точка с номером $n-1$~---~цвет 3, с номером $n$~---~цвет 2, с номером $n+1$~---~цвет 1, нет противоречий на стыке <<половинок>>). При этом, в любой паре $v_i$ и $v_{i+n}$ обе точки разного цвета. Таким образом, мы доказали, что для нечетных $n$ исходный граф можно раскрасить в 3 цвета, а, принимая во внимание замечание 1, мы доказали, что любой такой граф можно раскрасить в 3 цвета.\newline
Ответ: для любых $n$ \newline
\begin{center}
Раздел 11
\end{center}
{\bf 3)} Предположим обратное, тогда мы можем построить путь, за исключением какого-то кол-ва точек. Выберем одну из исключенных точек, тогда возможно несколько вариантов:
\begin{itemize}
\item Первый: если из этой точки нет выходящих граней, приходящих в точки нашего пути. Тогда, мы из последней точки нашего пути переходим в выбранную точку, таким образом получаем новый путь.
\item Второй: если есть хотя бы одна выходящая из точки грань, приходящая в точку нашего пути, и она приходит в последнюю точку нашего пути, тогда мы из предпоследней точки пути переходим в выбранную точку, а из нее~---~в последнюю, таким образом включаем нашу точку в путь.
\item Третий: если есть хотя бы одна выходящая из точки грань, приходящая в точку нашего пути и при этом она не приходит (ни одна из них) в последнюю точку пути, тогда алгоритм аналогично п.~1. 
\end{itemize}   
Таким образом, мы доказали, что если предположить, что нельзя построить такой путь, мы всегда можем собрать новый путь, добавляя в него непопавшие ранее точки $\rightarrow$ можем построить путь.	\newline
{\bf 4)} \newline
\begin{tikzpicture}
	[scale=.8,auto=left,every node/.style={circle,fill=blue!20}]
 	\node (no) 		at 	(0, 0)  	{о};
  	\node (nrub) 	at	(6, 0)		{руб};
  	\node (nn) 		at 	(4, -2)  	{н};
  	\node (nb) 		at 	(2, -4)		{б};
  	\node (ng) 		at 	(0, -6)		{ч};
  	\node (nt) 		at 	(4, -4) 	{т};
  	\node (nr) 		at 	(2, -6)		{рем};
  	\node (ngal) 	at 	(6, -2)		{гал};
  	\node (np) 		at 	(6, -6) 	{п};

  	\foreach \from/\to in {no/nrub, no/nn, no/nb, no/ng, nn/nt, nb/nt, nrub/ngal, nb/nr, nr/np, ngal/np}
    \draw[->] (\from) -- (\to);
\end{tikzpicture} \newline
Построив граф, нетрудно построить и правильную последовательность действий. \newline 
Например, очки$\rightarrow$часы$\rightarrow$брюки$\rightarrow$носки$\rightarrow$рубашка$\rightarrow$ремень$\rightarrow$галстук$\rightarrow$пиджак. \newline
Ответ: очки$\rightarrow$часы$\rightarrow$брюки$\rightarrow$носки$\rightarrow$рубашка$\rightarrow$ремень$\rightarrow$галстук$\rightarrow$пиджак \newline
{\bf 5а)} Выстроим граф городов: поместим один город вверху, один внизу, остальные между ними в линию. Из нижнего города все дороги исходящие, в верхний все дороги входящие. В любой центральный город входят дороги из нижнего города и из всех городов левее его, исходят дороги в верхний город и во все города правее его. \newline
{\bf 5б)}
\begin{itemize}
\item Предположим, что не существует города, в который нет входящих дорог, тогда в каждый город есть хотя бы одна входящая дорога. Тогда, возвращаясь в обратную сторону по таким дорогам, мы рано или поздно получим цикл (т.~к. кол-во вершин графа конечно) $\rightarrow$ противоречие с условием злого волшебника.
\item Предположим, что не существует города, из которого нет выходящих дорого, тогда из каждого города есть выходящая дорога. Тогда, переходя по этим дорогам, мы рано или поздно получим цикл (т.~к. кол-во вершин графа конечно) $\rightarrow$ противоречие с условием злого волшебника.
\end{itemize} 
{\bf 5в)} Предположим, он не единственен, тогда существует вершина, изменившая свою позицию в пути обхода города. Тогда, очевидно, существует другая вершина, которая в первом пути шла после выбранной нами вершины, а во втором~---~до (или наоборот). Тогда, мы начинаем идти по тому пути, где вторая вершина встречается позже (при этом проходим первую вершину), доходя до второй вершины, мы продолжаем идти уже по второму из путей. При этом, мы опять пройдем первую вершину
$\rightarrow$ возникнет цикл, что противоречит условию злого волшебника.
\end{document}