\section{Общий случай}

Теперь можно обратиться к общему случаю, т. е. к случаю произвольного электромагнитного поля, для которого справедливо дрейфовое приближение. Так как все дрейфы, которые были рассмотрены, в первом приближении независимы, то в общем случае их надо просто сложить. Таким путем для скорости сглаженного движения ведущего центра в пространстве, где не текут электрические токи, получаем
    
    \begin{equation}
        \textbf{V}  = v_{\parallel}\textbf{h} + \frac{c}{B^2}\textbf{[ЕВ]} + \frac{mc}{eBR}\bigg(v^2_{\parallel}+\frac{1}{2}v^2_{\perp}\bigg)\textbf{b}.
        \label{anvar}
    \end{equation}
    
Здесь произведены небольшие изменения в обозначениях. Под \textbf{h} мы теперь понимаем единичный вектор касательной к магнитной силовой линии, проходящей через ведущий центр, а не через саму частицу.Величины же $v_{\parallel}$ и $v_{\perp}$ означают усредненные скорости частицы вдоль этого нового вектора h и перпендикулярно к нему. Точно так же значения полей \textbf{B} и \textbf{E} мы берем в точке нахождения ведущего центра, а не частицы. Такая замена совершенно не затрагивает все слагаемые
в правой части формулы \ref{anvar}, за исключением первого, так как она меняет эти слагаемые только в первом или высшем порядке малости. Но эта замена существенна для слагаемого $v_{\parallel}\textbf{h}$, так как она нулевого порядка малости. Если бы сохранить прежний смысл вектора \textbf{h}, то в это слагаемое надо было бы ввести поправку первого порядка малости. Если же понимать \textbf{h} в новом смысле, как мы сделали, то такая поправка не нужна.
    
Итак, в сильном, но слабо неоднородном магнитном поле при наличии слабого электрического поля заряженная частица быстро вращается по ларморовской окружности. Центр ларморовской окружности движется вдоль магнитной силовой линии со скоростью $v_{\parallel}$ и испытывает дрейф перпендикулярно к магнитному полю. Дрейф вызывается электрическим полем и неоднородностями магнитного поля. Скорость электрического дрейфа определяется выражением $\dfrac{c}{B^2}[\textbf{Е В}]$.

Направление этого дрейфа не зависит от знака заряда частицы. Дрейф, вызываемый неоднородностями магнитного поля, происходит в направлении бинормали к магнитной силовой линии, причем положительно заряженные частицы дрейфуют в положительном направлении бинормали, а отрицательно заряженные~---~в противоположном направлении. Скорость этого <<магнитного>> дрейфа определяется выражением	

\begin{equation*}
    \frac{mc}{eBR}\bigg(v^2_{\parallel}+\frac{1}{2}v^2_{\perp}\bigg)\textbf{b}.	
\end{equation*}
