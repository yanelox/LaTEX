\section{Введение} 

В общем случае, когда магнитное и электрическое поля неоднородны и меняются во времени, движение частицы приобретает весьма сложный и запутанный характер. Проинтегрировать уравнения движения в аналитической форме в этом случае не удается. Для расчета движения приходится обращаться к сложным и утомительным численным методам. Есть, однако, случай, когда можно нарисовать сравнительно 
простую и обозримую картину движения, не обращаясь к численным методам расчета. Это будет тогда, когда магнитное поле сильное, а его изменения в пространстве и во времени происходят медленно. На магнитное поле может накладываться электрическое, но оно должно быть слабым по сравнению с магнитным. При этих условиях задачу можно приближенно решать по методу последовательных приближений.

В нулевом приближении полностью пренебрегают электрическим 
полем, а также пространственно-временными неоднородностями магнитного поля. Движение частицы представляется как быстрое вращение по ларморовскому кружку, центр которого перемещается вдоль 
магнитной силовой линии. Электрическое поле и пространственно- 
временные неоднородности магнитного поля учитываются в первом 
приближении. Они проявляются в том, что центр ларморовского кружка получает дополнительное медленное движение. Такое движение называется \textit{дрейфом}, а центр самого ларморовского кружка~---~\textit{ведущим центром} частицы. Параметры движения~---~циклотронная частота $\omega$, радиус ларморовского кружка $\rho$, продольная $\upsilon_{\parallel}$ и поперечная $\upsilon_{\perp}$ скорости частицы при этом будут медленно меняться. Медленность означает, 
что за циклотронный период $Т = 2\pi/\omega$ изменения этих параметров 
будут малы по сравнению со значениями самих параметров. Для этого 
необходимо, чтобы на протяжении ларморовского кружка и в течение циклотронного периода магнитное поле изменялось мало. Так как магнитное поле предполагается сильным, то размеры ларморовского 
кружка будут малы. Быстрые вращения по такому кружку часто не представляют интереса. Чтобы их исключить, достаточно усреднить движение частицы по временам порядка циклотронного периода. Тогда 
вместо движения самой частицы останется усредненное, или сглаженное, движение ее ведущего центра. Теория, рассматривающая движение частицы в такой постановке, называется \textit{дрейфовой}. За последние два-три десятилетия дрейфовая теория получила многочисленные применения при анализе движения космических частиц в межзвездных 
и межпланетных магнитных полях, а также в различного рода магнитных ловушках, предназначенных для удержания и нагрева плазмы 
с целью получения управляемых термоядерных реакций. Ниже в упрощенной форме излагаются основы дрейфовой теории и ее простейшие 
результаты. 
 
\section{Постановка задачи}
Задача дрейфовой теории~---~определить скорость плавного движения ведущего центра, обусловленного электрическим полем и пространственно~--~временными неоднородностями магнитного поля. В силу 
уравнения Максвелла: $ \partial\textbf{B}/\partial t = -c ~rot \textbf{E}$, временные неоднородности магнитного поля можно исключить, выразив их через соответствующие пространственные неоднородности поля электрического. Магнитное поле $\textbf{В}$, как уже сказано, предполагается сильным. Величины, пропорциональные $\textbf{В}$, считаются величинами \textit{нулевого порядка}. Это самые большие величины, но в дрейфовую теорию они не входят, так как проявляются только в быстрых вращениях по циклотронным окружностям, которые выпадают в результате усреднения по циклотронным периодам. Члены, пропорциональные электрическому полю и первым пространственным производным магнитного поля, считаются величинами \textit{первого порядка малости}. Влиянием первых и высших производных вектора $\textbf{E}$, вторых и высших производных вектора \textbf{В} будем пренебрегать. В этом приближении скорость плавного движения ведущего центра будет представляться линейной функцией напряженности электрического поля $\textbf{E}$ и первых пространственных производных вектора \textbf{В}. Понятно, что в принятом приближении все слагаемые этой линейной функции независимы и могут быть вычислены независимо друг от друга. Изменение магнитного поля в пространстве складывается из изменения его по абсолютной величине и из изменения по направлению. В соответствии с этим дрейфовое движение ведущего центра можно разложить на три движения:

\begin{enumerate}
    \item Дрейф под действием электрического поля, или электрический дрейф.
    \item Дрейф, вызванный изменениями магнитного поля только по абсолютной величине.
    \item Дрейф, вызванный изменениями магнитного поля только по направлению (т.е. искривлением магнитных силовых линий).
\end{enumerate}
