\documentclass[a4paper, 12pt]{article}

\usepackage{geometry}
\geometry{left=2cm, right=2cm, top=2cm, bottom=2cm}

\usepackage{cmap}
\usepackage{mathtext} 
\usepackage[T2A]{fontenc}
\usepackage[utf8]{inputenc}
\usepackage[english,russian]{babel}	

\usepackage{amsfonts,amssymb,amsthm,mathtools}
\usepackage{amsmath}
\usepackage{icomma} 

\usepackage{graphicx} 
\graphicspath{{picturies/}}
\usepackage{wrapfig}

\usepackage{array,tabularx,tabulary,booktabs}
\usepackage{longtable}
\usepackage{multirow}

\usepackage{caption}
\captionsetup{labelsep=period}

\renewcommand{\phi}{\varphi}
\newcommand{\eps}{\varepsilon}
\newcommand{\parag}[1]{\paragraph*{#1:}}

\newcounter{Points}
\setcounter{Points}{1}
\newcommand{\point}{\arabic{Points}. \addtocounter{Points}{1}}

\begin{document}

    Далее найдем диэлектрическую проницаемость для волны. Как и ранее, считаем, что волна распространяется перпендикулярно полю. Рассмотрим 2 случая:\\
    
    \begin{enumerate}
        \item Волна поляризована вдоль $E_0$
              Это значит, что $E_z ~\neq~ 0,~ E_x ~=~ E_y ~=~ 0,~ p_z ~=~ \beta E_z s_z^2$
              Вектор поляризации среды имеет вид:
              \begin{equation*}
                P_z ~=~ \int p_z dN ~=~ \dfrac{1}{4 \pi} N \beta E_z \int_0^{\pi} \int_0^{2 \pi} \left( 1 + \dfrac{a}{2} \left( s_z^2 - \dfrac{1}{3} \right) \right) s_z^2 \sin \theta d \theta d \phi ~=~
              \end{equation*}

              \begin{equation*}
                ~=~ \dfrac{1}{3} N \beta E_z \left( 1 + \dfrac{2}{15} a \right) ~=~ \alpha_{zz} E_z
              \end{equation*}

              Учитывая, что $E_z ~=~ 4 \pi P_z ~=~ \eps_{zz} E_z$, получаем
              \[
                  \eps_{zz} ~=~ 1 + 4 \pi \alpha_{zz} ~=~ 1 + \dfrac{4 \pi}{3} N \beta + \dfrac{8 \pi}{45} N \beta \alpha
              \]\\

              Эта компонента имеет вид $\eps_{zz} ~=~ \eps_{zz}^0 + \eps_{zz}^2 E_0^2$

              Тогда для показателя преломления имеем $n_z ~=~ \sqrt{\eps_{zz}} ~=~ n_z^0 + n_z^2 E_0^2$, где $n_z^0 ~=~ 1 + \dfrac{2 \pi}{3}N \beta,~ n_z^2 = \dfrac{4 \pi}{45} \dfrac{N \beta^2}{k T}$

        \item Теперь пусть поле $E$ перпендикулярно полю в конденсаторе и направлено по оси $x$. Тогда:
        
              \[
                  \begin{cases}
                      p_x ~=~ \beta E_x s_x^2 \\
                      p_y ~=~ \beta E_x s_x s_y \\
                      p_z ~=~ \beta E_x s_x s_z
                  \end{cases}
              \]

              Аналогично

              \[
                  P_x ~=~ \int p_x dN ~=~ \dfrac{1}{4 \pi} N \beta E_x \int_0^{\pi} \int_0^{2 \pi} \left( 1 + \dfrac{a}{2} \left( s_z^2 - \dfrac{1}{3} \right) \right) s_x^2 \sin \theta d \theta d \phi ~=~
              \]

              \[
                  ~=~ \dfrac{1}{3} N \beta E_x \left( 1 - \dfrac{1}{15} a \right)
              \]

              \[
                  P_y = \int p_y dN ~=~ \dfrac{1}{4 \pi} N \beta E_x \int_0^{\pi} \int_0^{2 \pi} \left( 1 + \dfrac{a}{2} \left( s_z^2 - \dfrac{1}{3} \right) \right) s_x s_y \sin \theta d \theta d \phi ~=~ 0
              \]

              \[
                  P_z ~=~ \int p_z dN ~=~ \dfrac{1}{4 \pi} N \beta E_x \int_0^{\pi} \int_0^{2 \pi} \left( 1 + \dfrac{a}{2} \left( s_z^2 - \dfrac{1}{3} \right) \right) s_x s_z \sin \theta d \theta d \phi ~=~ 0
              \]\\

            Учитывая, что $E_x + 4 \pi P_x ~=~ \eps_{xx} E_x$ получаем:
            
            \[
                \eps_{xx} ~=~ 1 + 4 \pi \alpha_{xx} ~=~ 1 + \dfrac{4 \pi}{3} N \beta - \dfrac{4 \pi}{45} N \beta \alpha
            \]\\

            Тогда для показателя преломления имеем $n_x ~=~ \sqrt{\eps_{xx}} ~=~ n_x^0 + n_x^2 E_0^2$, где $n_x^0 ~=~ 1 = \dfrac{2 \pi}{3} N \beta$, $n_x^2 ~=~ \dfrac{-2 \pi}{45} \dfrac{N \beta^2}{k T}$

            Ясно, что если бы мы направили поле $E$ по оси $y$, то аналогичным образом получили бы подобные формулы:\\

            \[
                P_x ~=~ 0,~ P_y ~=~ \dfrac{1}{3} N \beta E_y \left( 1 - \dfrac{1}{15}a \right),~ P_z ~=~ 0
            \]\\

            а также

            \[
                \eps_yy ~=~ 1 + 4 \pi \alpha_{xx} ~=~ 1 + \dfrac{4 \pi}{3} N \beta - \dfrac{4 \pi}{45} N \beta \alpha
            \]\\

            Полученный результат означает, что выбранные оси являются главными, и тензор проницаемости в них приводится к диагональному виду.
    \end{enumerate}

    \[
        \eps ~=~
        \begin{pmatrix}
        \eps_{xx} & 0 & 0 \\
        0 & \eps_{yy} & 0 \\
        0 & 0 & \eps_{zz}
        \end{pmatrix}
    \]\\

    Причем $\eps_{xx} ~=~ \eps_{yy} ~\neq~ \eps_{zz}$~---~как и в одноосном кристалле.\\

    Это означает, что величина $n_x ~=~ \sqrt{\eps_{xx}}$ имеет смысл показателя преломления для обыкновенного луча, а $n_z ~=~ \sqrt{\eps_{zz}}$~---~необыкновенного.\\

    Учитывая, что в отсутствие внешнего поля показатель преломления среды есть:

    \[
        n ~=~ n^0_x ~=~ n^0_z ~=~ 1 + \dfrac{2 \pi}{3}N \beta
    \]\\

    можно записать формулы в виде:

    \[
        n_e - n ~=~ \dfrac{4 \pi}{45} N \beta \alpha,~ n_0 - n ~=~ \dfrac{-2 \pi}{45} N \beta \alpha
    \]\\

    или

    \[
        \dfrac{n_e - n}{n_0 - n} = -2
    \]\\

    Это равенство хорошо выполняется для большинства веществ.\\

    Далее находим разность $n_e - n_0 ~=~ \dfrac{2 \pi}{15} N \beta \alpha ~=~ \dfrac{n - 1}{5} \dfrac{\beta E_0^2}{k T}$\\

    Получили выражение для постоянной Керра:

    \[
        B~=~\dfrac{n-1}{5 \lambda_0 k T} \beta
    \]\\

    Теория Ланжевена имеет небольшую проблему. Постоянная Керра получается всегда положительная не только для полностью анизотропных молекул, но и для молекул с произвольным тензором поляризуемости. Эта проблема была устранена в 1916 году Борном, который распространил ее на полярные молекулы со значительными постоянными дипольными моментами, направления которых могут не совпадать с направлениями наибольшей поляризуемости молекул. Так как собственный момент велик по сравнению с индуцируемыми моментами, то ориентация в таком случае будет определяться именно собственными моментами. Направление наибольшей поляризации среды может составлять с ним значительный угол, и если их направления взаимно перпендикулярны, то постоянная Керра считается отрицательной.

    Научно-техническое применение эффекта Керра основано на том, что явление имеет чрезвычайно быстрое время установления и исчезновения. Это обусловлено быстрым процессом поляризации и поворота молекул во внешнем поле – порядка 10-9 секунд. На основе эффекта Керра были придуманы быстродействующие затворы и модуляторы света, применяемые в лазерной технике для управления режима работы лазеров.
\end{document}