\documentclass[12pt,a4paper,fleqn]{article}
\usepackage[utf8]{inputenc}
\usepackage{amssymb, amsmath, multicol}
\usepackage[russian]{babel}
\usepackage{graphicx}
\usepackage[shortcuts,cyremdash]{extdash}
\usepackage{wrapfig}
\usepackage{floatflt}
\usepackage{lipsum}
\usepackage{concmath}
\usepackage{euler}
\usepackage{tikz}  
\usetikzlibrary{graphs}

\oddsidemargin=-15.4mm
\textwidth=190mm
\headheight=-32.4mm
\textheight=277mm
\tolerance=100
\parindent=0pt
\parskip=8pt
\pagestyle{empty}
\renewcommand{\tg}{\mathop{\mathrm{tg}}\nolimits}
\renewcommand{\ctg}{\mathop{\mathrm{ctg}}\nolimits}
\renewcommand{\arctan}{\mathop{\mathrm{arctg}}\nolimits}
\newcommand{\divisible}{\mathop{\raisebox{-2pt}{\vdots}}}

\begin{document}
{\bf 1)} Построим полный граф из 7 вершин, тогда в нем будет $\frac{7\cdot6}{2} = 21$. Добавим к нему еще одну вершину степени 1, т.е. с одним ребром, тогда в нашем графе станет $22$ ребра, причем $22$~---~максимально возможное кол-во ребер в заданном графе => т.к. $22<23$, ответ: нет. \newline

{\bf 2)} \newline
\begin{tikzpicture}
	[scale=.8,auto=left,every node/.style={circle,fill=blue!20}]
 	\node (n1) at (11, 1)  	{1};
  	\node (n2) at (13, 1)  	{2};
  	\node (n3) at (25, 3)  	{3};
  	\node (n4) at (15, 7)  	{4};
  	\node (n5) at (13, 9)  	{5};
  	\node (n6) at (21, 9)  	{6};
  	\node (n7) at (9,  7)  	{7};
  	\node (n8) at (9,  5)  	{8};
  	\node (n9) at (19,  3)  {9};

  	\foreach \from/\to in {n1/n2, n1/n5, n1/n8, n2/n4, n2/n7, n3/n6, n3/n9, n4/n5, n4/n8, n5/n7, n6/n9, n7/n8}
    \draw (\from) -- (\to);
\end{tikzpicture} \newline
Построив граф (наличие ребра означает наличие авиалинии между городами), нетрудно заметить, что города 3, 6, 9 образуют "замнкнутую систему", т.е. мы можем перемещаться только между ними, следовательно, в город 9 нельзя попасть из города 1. \newline

{\bf 3)} \newline
\begin{tikzpicture}
	[scale=.8,auto=left,every node/.style={circle,fill=blue!20}]
	\node (n1) at (0, 0)	{};
	\node (n2) at (3, 0)	{};
	\node (n3) at (6, 0)	{};
	\node (n4) at (10, 0)	{};
	\node (n5) at (8, -3)	{};
	
	\foreach \from/\to in {n1/n2, n3/n4, n4/n5, n5/n3}
	\draw (\from) -- (\to);
\end{tikzpicture} \newline
(+ бесконечное мн-во графов с n ребрами, где все ребра исходят из одной точки)\newline
Чтобы доказать, что других не существует, рассмотрим такой граф из n ребер. Выберем любое ребро, тогда она имеет общую точку с каждым => либо из точки 1 взятого ребра, либо из точки 2 взятого ребра исходят все наши ребра. Если из точки 1 выходит 0 ребер, то из точки 2 очевидно может выходить сколько угодно ребер. Если из точки 1 выходит 1 ребро, то из точки 2 выходит максимум одно ребро, т.к. иначе, чтобы связать эти ребра с ребром, выходящим из точки 1, мы получим "параллельные" ребра, которых в графе не существует. И, соответственно, если из точки 1 выходит больше одного ребра, то из точки 2 выходит 0 ребер. Кроме того, для случая когда из точек 1 и 2 выходит по однмоу ребру \newline

{\bf 4)} Выделим одну вершину (обозначим $A$), тогда она связана с 201 элементом (обозначим их за мн-во $X$), остается еще 198 (обозначим $Y$). Выбираем один элемент из $X$, тогда он тоже связан с 201 элементом. Т.к. он уже связан с $A$, остается 200. Т.к. 198 < 200, у нас выбранный элемент мн-ва $X$ связан еще как минимум с 2-мя элементами $X$, следовательно, возникнет цикл длины 3 (из 2-х элементов $X$ и элемента $A$). \newline

{\bf 5)} Нет, неверно, построим контр-пример: \newline
\begin{tikzpicture}
	[scale=.8,auto=left,every node/.style={circle,fill=blue!20}]
	\node (n1) at (0, 0)	{1};
	\node (n2) at (0, 2)	{2};
	\node (n3) at (2, 0)	{3};
	\node (n4) at (2, 2)	{4};
	
	\foreach \from/\to in {n1/n2, n2/n4, n4/n3, n3/n1}
	\draw (\from) -- (\to);
\end{tikzpicture} \newline
Тогда, $H1 = \{1, 2, 4\}, H2 = \{1, 3, 4\}$ - связные графы. Но, $H1 \cap H2 = \{1, 4\}$ - несвязный граф. \newline

{\bf 6)} Выберем один город ($A$), мн-во городов с которым он связан ($X$, мы расматриваем крайний случай, поэтому $|X| = 7$), и оставшееся мн-во городов ($Y$, $|Y| = 7$). Очевидно, что из любого города мн-ва $X$ и из $A$ можно попасть в любой город $X$ и в $A$. Теперь рассмотрим произвольный город из $Y$. Если он соединен со всеми городами $Y$, то задача решена. Если он не связан хотя бы с одним городом, тогда мы рассматриваем каждый такой город. В мн-ве $Y$ он может быть связан только с 5-ю городами, поэтому остается еще 2, поэтому он связан либо с $A$, либо с городом из $X$. Таким образом, из любого города можно попасть в любой. \newline

{\bf 7)} На языке графов: докажите, что в любом графе найдутся две вершины с одинаковой степенью. Чтобы докзазать, заметим, что в любом графе можно выделить компоненту связности с кол-вом вершин >1 (в ином случае, все вершины не связаны с друг другом, т.е. есть как 2 человека, которые ни с кем не знакомы, ч.т.д.). В таком, случае докажем, что в люой компоненте связности (т.е. в связном графе) найдется две вершины с одинаковой степенью. Предположим обратное, тогда у нас $n$ вершин, каждой из которых мы должны задать уникальное значение из мн-ва $X = \{1, 2, ..., n - 1\}$ (т.к. граф связный, минимальная степень вершины~---~$1$, а т.к. в нем $n$ вершин~---~максимальная степень это $n - 1$. Противоречие заключается в том, что в мн-ве $X$ у нас всего $n-1$ элемент, соответсвенно мы не можем выделить из него $n$ различных чисел никаким образом. Отсюда, в любом связном графе можно найти как минимум две вершины с одинаковой степенью, а следовательно и в любом графе можно найти две такие вершины, ч.т.д. \newline

{\bf 8)} Пусть, в исходном графе $n$ вершин, тогда если он является графом-путем, в нем $n-1$ ребро, а если он~---~граф-цикл, то в нем $n$ ребер. В полном графе, очевидно, $\frac{n(n-1)}{2}$ ребер. Рассмотрим три случая:
\begin{itemize}
\item Дополнением графа-пути является граф-путь: $\frac{n(n-1)}{2} = n - 1 + n - 1$ => \newline
$(n-4)(n-1) = 0$ => $n = 1, 4$ => \newline
\item Дополнением графа-цикла является граф-цикл: $\frac{n(n-1)}{2} = n + n$ => \newline
$n(n-5) = 0$ => $n = 5$
\item Дополнением графа-цикла является граф-путь: $\frac{n(n-1)}{2} = n + n - 1$ => \newline
$n^2 -5n + 2 = 0$~---~не имеет натуральных корней 
\end{itemize} 
ОТВЕТ: Дополнением графа-пути с кол-вом вершин 1, 4 является граф-путь, дополнением графа-цикла с кол-вом вершин 5 является граф-цикл, не существует графов-циклов, дополнением которых является граф-путь.
\end{document}