\documentclass[12pt,a4paper,fleqn]{article}
\usepackage[utf8]{inputenc}
\usepackage{amssymb, amsmath, multicol}
\usepackage[russian]{babel}
\usepackage{graphicx}
\usepackage[shortcuts,cyremdash]{extdash}
\usepackage{wrapfig}
\usepackage{floatflt}
\usepackage{lipsum}
\usepackage{concmath}
\usepackage{euler}
\usepackage{tikz}  
\usetikzlibrary{graphs}

\oddsidemargin=-15.4mm
\textwidth=190mm
\headheight=-32.4mm
\textheight=277mm
\tolerance=100
\parindent=0pt
\parskip=8pt
\pagestyle{empty}
\renewcommand{\tg}{\mathop{\mathrm{tg}}\nolimits}
\renewcommand{\ctg}{\mathop{\mathrm{ctg}}\nolimits}
\renewcommand{\arctan}{\mathop{\mathrm{arctg}}\nolimits}
\newcommand{\divisible}{\mathop{\raisebox{-2pt}{\vdots}}}

\begin{document}
{\bf 1а)} Представим эту задач в виде набора шаров (шары это ступеньки). Тогда, будем устанавливать разделители между шарами, причем наличие разделителя означает шаг на тот шар (ступеньку), который слева от разделителя. Таким образом, у нас всегда существует разделитель слева от самого левого шара (т.к. мы всегда оказываемся на нижней площадке). Остается расставить $k$ разделителей по $n$ местам. \newline 
$\frac{n(n-1)...(n - (k - 1))}{k!} = \frac{n!}{(n-k)!k!}$ Очевидно, что это $C_n^k$. Тогда ответ, это сумма:\newline
$\sum_{k = 1}^n C_n^k = (1 + 1)^n = 2^n$ ( по биному Ньютона).\newline
Ответ: $2^n$ \newline \newline
{\bf 1б)} Предположим, что ни для одной из наших пар чисел не выполняется условие для разности. Это означает, что все наши числа имеют различные остатки при делении на $12$. Представим все наши числа в виде: $x_i = 12a_i + b_i$, где $b_i \in \{ 0, 1, 2, ..., 11\}$, и при этом все $b_i$ различны. \newline
Выберем некоторые числа $x_1$ и $x_2$ и перемножим их:\newline
$x_1x_2 = (12a_1 + b_1)(12a_2 + b_2) = 144a_1a_2 + 12a_1b_2 + 12a_2b_1 + b_1b_2$ \newline
Очевидно, что такое число делится на $12$ только в том случае, если $b_1b_2$ делится на $12$. Тогда, числа $b_1$ и $b_2$ являются одной из пар множества: $\{(2, 6), (3, 4), (3, 8), (4, 6), (4, 9), (6, 8), (8, 9)\}$, либо одно из них равно $0$. Попытаемся построить такое множество остатков, что среди них не будет нуля и ни возникнет ни одной из вышеперечисленных пар. \newline
Очевидно, можем включить в это множество числа $1, 5, 7, 10, 11$. Остается выбрать 4 числа из множества: $\{2, 3, 4, 6, 8, 9\}$. Если мы добавим все числа, кроме пары $(3, 4)$, то у нас возникнет пара $(2, 6)$. Eсли добавим число $3$, но не добавим число $4$, то у нас либо возникнет пара $(2, 6)$, либо, если мы исключим также одно из чисел такой пары, возникнет пара $(3, 8)$. Если добавим, число $4$, но не добавим число $3$, то у нас либо возникнет пара $(2, 6)$, либо, если мы исключим также одно из чисел такой пары, возникнет пара $(4, 9)$. Ну и, очевидно, что мы не можем одновременно включить пару $(3, 4)$ в наше множество.\newline
Таким образом, мы доказали, что невозможно выбрать множество остатков таким образом (при всех различных остатках), чтобы у нас не было пары чисел, произведение которых делится на 12. Из этого следует, что для любых 9 целых чисел выполняется условие о разности, либо, в случае его невыполнения, выполняется условие о произведении, ч.~т.~.д. \newline \newline
{\bf 2а)} Найдем кол-во расстановок, при котором хотя бы две из этих трех книг стоят рядом. 
\begin{itemize}
\item Если все три стоят рядом: $18 \cdot 3 \cdot 2 = 18 \cdot 6$~---~$18$ способов поставить три книги рядом в ряду из $20$ книг, $6$ способов переставить выбранные книги внутри тройки.
\item Если только две стоят рядом: $(2 \cdot 17 + 16 \cdot 17) \cdot 6= 17 \cdot 18 \cdot 6$. $17 \cdot 2$~---~когда две книги стоят вместе и при этом одна из них на краю полки ($17$ способов поставить оставшуюся книгу), $16 \cdot 17$~---~когда две книги стоят вместе и при этом не около края полки ($16$ способов поставить оставшуюся книгу), $6$ способов переставить книги внутри тройки.
\end{itemize}
Таким образом, $(18 + 17 \cdot 18) \cdot 6 = 18 \cdot 18 \cdot 6$. Кроме того, нужно домножить на $17!$, т.~к. остальные $17$ книг мы можем как-угодно расставить. Тогда, итоговое число $18 \cdot 6 \cdot 18!$. \newline
А т.~к. нам нужны все случаи, кроме нужныъ нам, мы должны вычесть: \newline
$20! - 18 \cdot 6 \cdot 18! = 272 \cdot 18!$ \newline
Ответ: $272 \cdot 18!$ \newline \newline
{\bf 2б)} Выставим все книги в ряд и начнем расставлять между ними $k$ разделителей (причем разделители могут стоять на $n + 1$  месте, если мы имеем $n$ книг). Тогда, кол-во способов, которыми мы можем расставить разделители: $\frac{(n + 1)n...(n + 1 - (k - 1))}{k!} = \frac{(n+1)!}{(n+1 - k)!k!} = C_{n+1}^k$. В нашем случае $5$ коробок и $20$ книг $\rightarrow$ $k = 4, n = 20$ \newline
Таким образом, получаем $C_{21}^4$. Кроме того, порядок книг, выставленных в ряд также может меняться, таким образом, получаем $C_{21}^4 \cdot 20!$ \newline
Ответ: $C_{21}^4 \cdot 20!$ \newline \newline
{\bf 2в)} Удалим все буквы и посчитаем кол-во всех возможных буквенных комбинаций из оставшихся букв: 
\begin{itemize}
\item Всего $18$ букв
\item Буква <<в>> повторяется $3$ раза, буквы <<е>>, <<с>> и <<т>> по 2 раза. Букв <<о>> - $5$.
\end{itemize} 
Тогда, число комбинаций без букв <<о>>: \newline
$\frac{18!}{3!2!2!2!} = \frac{18!}{48}$ \newline
Тогда, мы просто расставляем буквы <<o>> на $19$ позиций. Т.~к. букв $5$, число таких перестановок, очевидно, $C_{19}^5$ \newline
Таким образом, получаем число $\frac{18! C_{19}^5}{48}$ \newline
Ответ: $\frac{18! C_{19}^5}{48}$ \newline \newline
{\bf 3а)} Расположим пирожные в ряд и расставим $k$ разделителей на $n + 1$ место (т.~к. мы можем не купить ни одного пирожного какого-либо вида). Тогда, кол-во способов расставить эти разделители: $C_{n+1}^k$ ( из задачи 2б). Мы имеем $27$ пирожных и $4$ вида $\rightarrow$ $k = 3, n = 27 \rightarrow C_{28}^3$ \newline
Ответ: $C_{28}^3$ \newline \newline
{\bf 3b)} Запишем $x_i = y_i + i - 1$, (где $y_i \geqslant 1$) $\rightarrow y_1 + y_2 + 1 + y_3 + 2 + y_4 + 3 = 36 \rightarrow$ \newline
$y_1 + y_2 + y_3 + y_4 = 30$ \newline
Расположим $30$ ($n$) шаров в линию и поставим между ними $3$ разделителя ($k$), причем разделители стоят только между шарами ($n - 1$ позиция). Тогда, мы имеем:\newline
$\frac{(n-1)(n-2)...(n - 1 - (k - 1))}{k!} = \frac{(n-1)!}{(n - 1 - k)!k! = C_{}} = C_{n-1}^k = C_{29}^3$ \newline
Ответ: $C_{29}^3$ \newline \newline
{\bf 3c)} $2x_1 + 2x_2 + 2x_3 + 21x_4 + 2x_5 \leqslant 66$
\begin{itemize}
\item $x_4 = 1$. Тогда получаем: \newline
$2x_1 + 2x_2 + 2x_3 + 2x_5 \leqslant 45$ \newline
$x_1 + x_2 + x_3 + x_5 \leqslant 22.5$ \newline
$4 \leqslant x_1 + x_2 + x_3 + x_5 \leqslant 22$ (т.~к. числа натуральные) \newline
Решения этого неравенства можно представить в виде: \newline
$x_1 + x_2 + x_3 + x_5 = n, n \in \{4, 5, ..., 22\}$ \newline
Чтобы найти кол-во решений в таком уравнении, нарисуем $n$ шаров и расставим между ними $3$ разделителя (по $n - 1$ местам). Тогда кол-во способов сделать это и будет кол-вом решений уравнения:
$\frac{(n - 1)(n - 2)(n - 3)}{3!} = \frac{(n - 1)!}{(n - 4)! 3!}$ \newline
Тогда, кол-во решений неравенства: \newline
$\sum_{n = 4}^{22} \frac{(n - 1)!}{(n - 4)! 3!}$
\item $x_4 = 2$ Тогда: \newline
$2x_1 + 2x_2 + 2x_3 + 2x_5 \leqslant 24$ \newline
$4 \leqslant x_1 + x_2 + x_3 + x_5 \leqslant 12$ \newline
Кол-во решений аналогично случаю $x_4 = 1$, только $n \in \{4, 5, ..., 12\} \rightarrow$ \newline
$\sum_{n = 4}^{12} \frac{(n - 1)!}{(n - 4)! 3!}$
\item $x_4 \geqslant 3$ Тогда:
$2x_1 + 2x_2 + 2x_3 + 2x_5 \leqslant 66 - 21x_4 \leqslant 3$ не имеет решений в натуральных числах
\end{itemize}
Ответ: $\sum_{n = 4}^{22} \frac{(n - 1)!}{(n - 4)! 3!} + \sum_{n = 4}^{12} \frac{(n - 1)!}{(n - 4)! 3!}$ \newline \newline
{\bf 4a)} Рассмотрим кол-во способов, которыми можно разместить квадрат $k\times k$ внутри нашего квадрата $n\times n$. Очевидно, что проекции сторон на верхнюю и боковую тоже будут размером $k$, тогда мы можем двигать каждую из проекций вдоль соответстующей стороны, таким образом, получаем $n - k + 1$ позиций для каждой проекции $\rightarrow$ $(n - k + 1)^2$  способов размещения самого квадрата. Тогда, внутри квадрата $n \times n$ мы можем размещать квадраты размером до $n \times n$ включительно, т.~е. $k \in \{1,..., n\} \rightarrow$ ответом будет являться число $\sum_{k = 1}^n (n - k + 1)^2 = n^2 + (n - 1)^2 + ... + 2^2 + 1^2 = \frac{n(n + 1)(2n + 1)}{6}$ \newline
Ответ: $\frac{n(n + 1)(2n + 1)}{6}$ \newline\newline
{\bf 4b)} Аналогично пункту a, мы выбираем первую сторону (длиной $a$) из верхней грани, тогда способов ее выбрать $n - a + 1$, а т.~к. $a \in \{1,...,n\}$, то всего $\sum_{a = 1}^n (n - a + 1)$ способов. Аналогично, для второй стороны (длиной $b$). Тогда, всего способов выбрать прямоугольник: $\sum_{a, b = 1}^n ((n - a + 1)(n - b + 1)) = (\sum_{a = 1}^n (n - a + 1)) \cdot (\sum_{b = 1}^n (n - b + 1)) = \frac{n + 1}{2}n\frac{n + 1}{2}n = \frac{(n^2 + n)^2}{4}$ (по формуле суммы арфимитической прогрессии). \newline
Ответ: $\frac{(n^2 + n)^2}{4}$ \newline \newline
{\bf 4c)} Заметим, что для буквы Г любого вида можно построить прямоугольник вокруг нее, и причем каждому прямоугольнику соответствует $4$ Г. Соответственно, задача сводится к подсчету кол-ва прямоугольников, которых мы можем нарисовать, с линейными размерами $\geqslant2$. Тогда, воспользуемся формулой, полученной в пункте $b)$, немного изменив ее: \newline
$4 \cdot (\sum_{a = 2}^n (n - a + 1)) \cdot (\sum_{b = 2}^n (n - b + 1)) = 4 \cdot (\frac{1 + n - 1}{2}(n - 1))^2 = (n^2 - n)^2$ \newline
Ответ: $(n^2 - n)^2$
\end{document}