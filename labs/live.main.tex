\documentclass[a4paper, 12pt]{article}

\usepackage{cmap}
\usepackage{mathtext} 
\usepackage[T2A]{fontenc}
\usepackage[utf8]{inputenc}
\usepackage[english,russian]{babel}	

\usepackage{amsfonts,amssymb,amsthm,mathtools}
\usepackage{amsmath}
\usepackage{icomma} 

\usepackage{graphicx} 
\graphicspath{{Picturies/}}
\usepackage{wrapfig}

\usepackage{array,tabularx,tabulary,booktabs}
\usepackage{longtable}
\usepackage{multirow}

\usepackage{caption}
\captionsetup{labelsep=period}

\renewcommand{\phi}{\varphi}
\newcommand{\eps}{\varepsilon}
\newcommand{\parag}[1]{\paragraph*{#1:}}

\author{Радькин Кирилл Б01-005}
\title{3.2.3. Резонанс токов}
\date{13.09.21}

\graphicspath{{pictures/}}


\begin{document}


\usepackage{cmap}
\usepackage{mathtext} 
\usepackage[T2A]{fontenc}
\usepackage[utf8]{inputenc}
\usepackage[english,russian]{babel}	

\usepackage{amsfonts,amssymb,amsthm,mathtools}
\usepackage{amsmath}
\usepackage{icomma} 

\usepackage{graphicx} 
\graphicspath{{Picturies/}}
\usepackage{wrapfig}

\usepackage{array,tabularx,tabulary,booktabs}
\usepackage{longtable}
\usepackage{multirow}

\usepackage{caption}
\captionsetup{labelsep=period}

\renewcommand{\phi}{\varphi}
\newcommand{\eps}{\varepsilon}
\newcommand{\parag}[1]{\paragraph*{#1:}}

\author{Радькин Кирилл Б01-005}
\title{3.2.3. Резонанс токов}
\date{13.09.21}

\graphicspath{{pictures/}}


\begin{document}

\maketitle

\parag {Цель работы} изучение параллельной цепи переменного тока, наблюдение резонанса токов.

\parag{В работе используются} лабораторный автотрансформатор (ЛАТР), разделительный понижающий трансформатор, емкость, дроссель с переменной индуктивностью, три амперметра, вольтметр, реостат, электронный осциллограф, омметр, мост переменного тока.
\\\\
В работе изучается параллельный контур, одна из ветвей которого содержит индуктивность $L$, другая емкость $C$. Через $r_L$ обозначено активное сопротивление катушки, которое включает в себя как чисто оммическое сопротивление витков катушки, так и сопротивление, связанное с потерями энергии при перемагничиваниии сердечника катушки. Активным сопротивлением емкостной ветви контура можно пренебречь, т.к. используемый в работе конденсатор обладает малыми потерями.
\\\\


\end{document}

\end{document}
