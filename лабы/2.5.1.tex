\documentclass[12pt,a4paper,fleqn]{article}
\usepackage[utf8]{inputenc}
\usepackage{amssymb, amsmath, multicol}
\usepackage[russian]{babel}
\usepackage{graphicx}
\usepackage[shortcuts,cyremdash]{extdash}
\usepackage{wrapfig}
\usepackage{floatflt}
\usepackage{lipsum}
\usepackage{concmath}
\usepackage{euler}
\usepackage{tikz}  
\usetikzlibrary{graphs}

\oddsidemargin=-15.4mm
\textwidth=190mm
\headheight=-32.4mm     
\textheight=277mm
\tolerance=100
\parindent=0pt
\parskip=8pt
\pagestyle{empty}
\renewcommand{\tg}{\mathop{\mathrm{tg}}\nolimits}
\renewcommand{\ctg}{\mathop{\mathrm{ctg}}\nolimits}
\renewcommand{\arctan}{\mathop{\mathrm{arctg}}\nolimits}
\newcommand{\divisible}{\mathop{\raisebox{-2pt}{\vdots}}}

\graphicspath{{pictures/}}

\begin{document}
    \begin{center}
        Лабораторная работа 2.5.1
        \\
        "Измерение коэффициента поверхности натяжения жидкости"
        \\
        Б01-005 Радькин Кирилл
    \end{center}
    Цель работы:
    \begin{itemize}
        \item Измерение коэффициента поверхностного натяжения дистилированной воды при разной температуре с использованием известного коэффициента поверхностного натяжения спирта; 
        \item Определение полной поверхностной энергии и теплоты, необходимой для изотермического образования единицы поверхности жидкости
    \end{itemize}

    Экспериментальная установка: \\
    \includegraphics[scale=0.8]{image001_35.jpg}

    Ход работы:\\\\
    1)Убедимся в исправности установки\\\\
    2) Подберем частоту падения капель из аспиратора так, чтобы максимальное давление микроманометра не зависело от этой частоты. Для этого пузырьки не должны пробулькивать слишком часто (не чаще, чем $1$ пузырек в $5$ секунд)\\\\
    3) Измерим максимальное давление при пробулькивании пузырька (в спирте, при комнатной температуре $t = 21^{\circ}C$) и подсчитаем радиус капиляра. \\\\
    \begin{tabular}{c | c | c | c | c | c}
        $h$, дел. & $38$ & $39$ & $38$ & $39$ & $39$ \\ \hline
        $P$, Па & $74,3$ & $76,2$ & $74,3$ & $76,2$ & $76,2$ \\ 
    \end{tabular} \\\\
    $h$ - кол-во делений на микроманометре, $P$ - давление \\\\
    $P = c \cdot h \cdot \dfrac{\gamma_1}{\gamma_2} \cdot K \cdot 9.81$, где:\\
    \begin{itemize}
        \item $c$ = 1
        \item $K$ = 0.2 (коэффициент, зависящий от угла наклона)
        \item $\gamma_1 = 0.8066$ г/см$^3$~---~плотность залитого спирта
        \item $\gamma_2 = 0.8095$ г/см$^3$~---~плотность приборного спирта
    \end{itemize}
    $<P> = 75.4$ Па, $\sigma_p = \sqrt{\dfrac{1}{n(n-1)} \sum_{i=1}^n (P_i - <P>)^2} = 0.5$ Па $\rightarrow$ $<P> = 75.4 \pm 0.5$ Па \\\\
    4) Используя формулу Лапласа ($\Delta P = \dfrac{2 \sigma}{R}$) найдем радиус капиляра $R = \dfrac{2 \sigma}{<P>} = 0.59$ мм \\
    Сравним его с радиусом, измеренным с помощью микроскопа: $R_m = 0.6$ мм \\\\
    5) Перенесем (предварительно просушив) иглу в сосуд с водой. Измерим максимальное давление, когда игла лишь касается поверхности жидкости. \\\\
    \begin{tabular}{c | c | c | c | c | c}
        $h_1$, дел. & $101$ & $106$ & $107$ & $107$ & $108$ \\ \hline
        $P_1$, Па & $197.5$ & $207.2$ & $209.2$ & $209.2$ & $211.1$ \\
    \end{tabular} \\\\
    $<P_1> = 206.8 \pm 2.4$ Па\\\\
    6) Измерим $l_1 = 5.7$ см~---~расстояние от конца капиляра до некоторой части прибора.\\\\
    7) Утопим иглу до предела, но так, чтобы выходящие пузырьки не касались дна и снова измерим максимальное давление \\\\
    \begin{tabular}{c | c | c | c | c | c}
        $h_2$, дел. & $182$ & $182$ & $182$ & $182$ & $181$ \\ \hline
        $P_2$, Па & $355.8$ & $355.8$ & $355.8$ & $355.8$ & $353.8$ \\
    \end{tabular} \\\\
    $<P_2> = 355.4 \pm 0.4$ Па \\\\
    8) Снова измерим расстояние от конца капиляра до той же части прибора: $l_2 = 6.2$ см\\\\
    9) Подсчитаем $\Delta h = h_2 - h_1 = 1.5$ см. Сравним с $\Delta h'$, подсчитанным с помощью разницы давлений: $\Delta h' = \dfrac{<P_2> - <P_1>}{\rho g} = 1.5$ см \\\\
    10) Снимем зависимость $\sigma (t)$:\\\\
    \begin{tabular}{c | c | c | c | c | c | c | c}
        $t^{\circ}C$ & $27$ & $31$ & $35$ & $41$ & $45$ & $50$ & $55$\\ \hline
        $h,$ дел. & $180$ & $181$ & $177$ & $175$ & $172$ & $171$ & $168$\\ \hline
        $P_m,$ Па & $351.9$ & $353.8$ & $346.0$ & $342.1$ & $336.3$ & $334.3$ & $328.4$\\ \hline
        $\Delta P$, Па & $205.0$ & $207.0$ & $199.2$ & $195.3$ & $189.4$ & $187.4$ & $181.6$\\ \hline
        $\sigma$, Н/м & $0.062$ & $0.062$ & $0.060$ & $0.059$ & $0.057$ & $0.056$ & $0.054$\\ 
    \end{tabular} \\\\
    11) Методом наименьших квадратов вычисляем коэффициенты $k$ и $b$ в зависимости $\sigma = k \cdot t + b$: \\
    $k = (-2.7 \pm 0.2) \cdot 10^{-4}$ Н/м$^{\circ}$с; $b = (694 \pm 1) \cdot 10^{-4}$ Н/м \\\\
    \includegraphics[scale=0.8]{загруженное.png}\\\\
    Зеленый график~---~$\dfrac{U}{F} = \sigma - t \cdot \dfrac{d \sigma}{dt} = \sigma - kT = b$ (поверхностная энергия единицы площади поверхности)\\
    Синий график~---~$\sigma = k \cdot t + b$ (коэффициент поверхностного натяжения от температуры) \\ 
    Оранжевый график~---~$q = -t \cdot \dfrac{d \sigma}{dt} = -k \cdot t$ \\\\
    12) Вывод: данный эксперимент с достаточно большой точностью позволяет выявить линейную зависимость коэффициента поверхностного натяжения от температуры.
\end{document}