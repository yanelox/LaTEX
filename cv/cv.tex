\documentclass[a4paper, 12pt]{article}

\usepackage{geometry}
\geometry{left=2cm, right=2cm, top=2cm, bottom=2cm}

\usepackage{cmap}
\usepackage{mathtext} 
\usepackage[T2A]{fontenc}
\usepackage[utf8]{inputenc}
\usepackage[english,russian]{babel}	

\usepackage{amsfonts,amssymb,amsthm,mathtools}
\usepackage{amsmath}
\usepackage{icomma} 

\usepackage{graphicx} 
\graphicspath{{picturies/}}
\usepackage{wrapfig}

\usepackage{array,tabularx,tabulary,booktabs}
\usepackage{longtable}
\usepackage{multirow}

\usepackage{caption}
\captionsetup{labelsep=period}

\renewcommand{\phi}{\varphi}
\newcommand{\eps}{\varepsilon}
\newcommand{\parag}[1]{\paragraph*{#1:}}

\newcounter{Points}
\setcounter{Points}{1}
\newcommand{\point}{\arabic{Points}. \addtocounter{Points}{1}}
 
\begin{document}
\section*{\underline{Радькин Кирилл Алексеевич}}
\subsection*{\underline{Контакты:}}
EMAIL: radkin.ka@phystech.edu \\
TG: @jkoertp \\
GitHub: https://github.com/yanelox \\

\subsection*{\underline{О себе:}}

\begin{itemize}
    \item Во время учебы в школе участвовал в олимпиадах по физике, реже по информатике и математике. 
    \item В 7 классе проходил курс по Java (Samsung IT School), но уже мало что помню
    \item На первом курсе обучался в продвинутой группе по программированию, у Владимирова Константина Игоревича. 
    \item Также, на первом курсе проходил (не закончил) курс по C Intel ILab
    \item В данный момент являюсь студентом 2-го курса ФРКТ МФТИ
    \item Прохожу курс по C++ от Владимирова К.И. (Intel)
    \item Закончил (почти, т.к. курс завершился преждевременно из-за ухода Intel) курс MIPT~V по архитектуре процессора 
    \item Уровень владения английским: Pre Intermediate, довольно свободно говорю, могу разбираться в технической литературе (со словарем)
\end{itemize}

\subsection*{\underline{Навыки:}}
\begin{itemize}
    \item C/C++ \\\\
    Проекты на C:
    \begin{itemize}
        \item Акинатор, калькулятор (рекурсивный спуск)
        \item Stack, hash-table
        \item Онегин (задачи сортировки текста)
        \item Небольшие задания на жадные алгоритмы и динамическое программирование
    \end{itemize}
    Проекты на C++:
    \begin{itemize}
        \item ARC-кэш, идеальный кэш (алгоитм Белади), сравнение эффективности
        \item Балансировка AST-дерева, поиск определенных элементов внутри
        \item Поиск пересечений между плоскими треугольниками в пространстве, отрисовка с помощью VulkanAPI
        \item Обход в ширину/глубину графа, проверка на двудольность
    \end{itemize}

    \item Базовые знания Python (начинал проходить курс от ФПМИ МФТИ DeepLearning School)
\end{itemize}

\subsection*{Хобби:}
\begin{itemize}
    \item Читаю литературу, в последнее время чаще техническую, ранее увлекался Стивеном Кингом, Артуром Конан Дойл, Агатой Кристи, Рэй Брэберри
    \item Люблю играть в баскетбол, иногда даже получается
\end{itemize}
\end{document}